\documentclass{hnureport}
\usepackage{xeCJK} %调用 xeCJK 宏包
\setCJKmainfont{SimSun} %设置 CJK 主字体为 SimSun (宋体

\renewcommand{\abstractname}{\large Abstract\\}%重定义摘要二字的大小
\begin{document}
\begin{titlepage}
    \clearpage\thispagestyle{empty}
    \centering
    \vspace{1cm}
    % Titles
    {\
        \textsc{Mao Zedong thought and the theoretical system of socialism with Chinese characteristics}
    }
    \vspace{2.5cm}
    
    \rule{\linewidth}{2mm} \\[0.5cm]
    { \Huge \bfseries How To Treat The Trade Friction Between CHINA And USA\\[0.5cm]
        如何看待中美贸易摩擦}\\[0.5cm]
    \rule{\linewidth}{0.6mm} \\[1.5cm]
    
    \hspace{2cm}
    \begin{tabular}{l p{5cm}}
        \textbf{Title} & 如何看待中美贸易摩擦 \\[10pt]
        \textbf{Member} & 梁帮博 \quad 屈德林 \quad 陶安源 \quad 王俊瑜 \quad 支伟 \\[10pt]
        \textbf{Class} & 计算机科学与技术 1805 \\[10pt]
        \textbf{Department} & CSEE \\[10pt]
        \textbf{Date} & \today \\            
    \end{tabular}
    
    % logo
    \vfill
    \centering \includegraphics[height=3.5cm]{Figure/HNUlogo.pdf}\\ % light logo
    \centering \includegraphics[scale=0.3]{Figure/logo_slogan.png}
    \vspace{0.5cm}

    \pagebreak
\end{titlepage}

\begin{abstract}




\end{abstract}

\noindent\textbf{关键词:} 中美贸易摩擦 \quad  中美关系 \quad 湖南大学 \LaTeX
\newpage%摘要

\thispagestyle{empty}
\tableofcontents
\newpage 
\setcounter{page}{1}

\section{课题重述与背景}
\subsection{课题重述}
湖南大学电气与信息工程学院,起源可追溯到1921年湖南公立工业专门学校的电机科。1926年湖南大学定名时,电机科改为电机系。1953年电机系被调整到华中工学院,1958年恢复。1980年电机系与电子工程系合并为电气工程系,1999年成立电气与信息工程学院。 该院有电气工程系、自动化系、电子信息工程系、仪器科学系、电工电子技术系和1个实验中心。全院教职工160多人,学院拥有全职院士1人(罗安),双聘院士1人,国家“百千万人才工程”第一、二层次人选2人(王耀南,李树涛),国家杰出青年科学基金获得者2人(曹一家、李树涛),教育部新世纪优秀人才8人。在籍研究生、本科生近3000人。
\subsection{课题背景}
湖南大学数学学院是湖南大学历史最悠久的院系之一,它源始于清末1908年湖南优级师范学堂(湖南大学的前身)设立的数学本科,从1924年的数理系、1933年的数学系、1959年的数学力学系、1985年的应用数学系,2000年的数学与计量经济学院到2019年的数学学院,已历经了近百年的变革与发展,硕果累累。学院下设应用数学、基础数学、信息与计算科学、概率统计、计量经济5个系,应用数学、运筹学与控制论和高等数学3个研究所,科学与工程计算实验室和资料室。学院拥有数学学科博士后科研流动站,应用数学、计算数学2个博士点,应用数学、基础数学、计算数学、概率统计、运筹学与控制论5个硕士点,数学与应用数学、信息与计算科学2个本科专业。其中,应用数学学科是湖南省“十五”重点建设学科,数学与应用数学专业是湖南省重点专业,《高等数学》课程是国家级精品课程。



\section{课题结果}
学院院长为王耀南教授,曹一家教授为学校现任副校长,王耀南、罗安2位教授为“长沙市十大知识产权创造导师”。
学院科研基础好、学科综合优势强,已形成电力系统自动化、电机传动、工业自动化、测控与仪器、电子信息与通信工程、智能控制与图象处理、新型输配电新技术、高电能质量输配电理论和技术、大规模集成电路故障诊断等特色和优势研究方向。学院先后承担了国家 “九五”、“十五” 、“十一五”攻关项目,国家 “863”、国际合作重点项目,国家自然科学基金,国家创新基金,博士点基金和部省科研基金项目300多项,其他横向科研课题350余项,国家发明特等奖1项,国家科技发明三等奖1项,部、省科技进步一、二等奖和其他奖项120余项。07年到帐科研经费1500多万元。近年来国家科技进步二等奖有:王耀南4项:智能图像信息处理方法及其在工业系统中的应用;高速灌装生产线智能检测分拣成套装备研制及其推广应用等。罗安2项:电能质量先进控制方法及工程应用;大型企业综合电气节能关键技术及应用;国家技术发明二等奖有:罗安一项:冶金特种大功率电源系统关键技术与装备及其应用。近2年获得IEEE第三大区唯一杰出工程师年度大奖1人,全国优秀科技工作者1人。

电气与信息工程学院传承“爱国务实、经世致用”的优良传统,积极参与“推进富民强省、争当五一先锋”竞赛,在教学、科研、学科建设等领域喜获全面丰收。一是人才培养卓有成效。学院确立人才培养的中心地位,创新人才培养成绩突出,近三年指导博士生获,“全国百篇优秀博士学位论文”获得者1人,省优秀博士论文5篇、省优秀硕士论文12篇、在全国“挑战杯” 大学生创业大赛上荣获金奖3人、全国大学生智能车竞赛上荣获一等奖6项;教学研究和教学改革成果显著,先后承担国家级教改项目1项、省部级教改项目5项。获国家级教学成果二等奖1项,省级教学成果一等奖1项,二等奖3项;工业自动化专业被评为国家特色专业,电气工程及其自动化专业被列为湖南省重点专业。二是科学研究硕果累累。学院围绕国家经济和国防建设重大需求,集中力量,以国家 “973”项目、“863”计划、国家重大科技专项、国家科技支撑计划、国家自然基金重点项目、湖南省重大专项为突破口,积极服务于国家和地方经济,实现了产学研的有机结合。形成了多个高水平科研团队,营造了良好的科研环境。近三年新增科研项目170项,科研项目到帐总经费8274万元,出版教材、学术专著15部,获得发明专利47项、软件著作权51项、实用新型专利33项,学术论文被三大检索收录479篇,荣获各类科技奖励共34项,其中国家科技进步二等奖4项,省部级奖励19项。三是学科建设成绩突出。学院已拥有十分完备的学科体系,拥有“控制理论与控制工程”国家重点学科,“电工理论与新技术”国家重点培育学科;拥有1个机器人感知与控制技术国家工程实验室,1个电能变换与控制国家工程技术研究中心,1个电力驱动与伺服技术国防重点学科实验室,2个教育部工程研究中心,1个教育部重点实验室,1个国防技术重点实验室,2个湖南省重点实验室,2个机械工业联合会重点实验室,2个博士后科研流动站,2个一级学科博士点,11个二级学科博士点,17个硕士点。四是综合管理井然有序。学院形成了“拼搏、奉献、和谐、快乐”的学院文化,制度健全、管理完善,院务公开、民主决策、勤政廉政,注重依靠教职工共商发展大计,凝心聚力,有关学院建设和发展的重大决策和涉及教职工切身利益的重大事项,都必须通过教职工代表大会的民主决策。学院呈现出持续快速发展的良好局面,党建与思政工作连续六年被评为学校优秀单位;部门工会连续七年被评为先进集体,2010年被授予湖南大学先进教职工之家称号,2011年被湖南省总工会授予“湖南省五一先锋集体”荣誉称号,2010年湖南省教育系统唯一获得该项荣誉的单位。

\input{reference.tex}

\appendix

\input{appendix.tex}

\end{document}
