\section{课题结果}
学院院长为王耀南教授,曹一家教授为学校现任副校长,王耀南、罗安2位教授为“长沙市十大知识产权创造导师”。
学院科研基础好、学科综合优势强,已形成电力系统自动化、电机传动、工业自动化、测控与仪器、电子信息与通信工程、智能控制与图象处理、新型输配电新技术、高电能质量输配电理论和技术、大规模集成电路故障诊断等特色和优势研究方向。学院先后承担了国家 “九五”、“十五” 、“十一五”攻关项目,国家 “863”、国际合作重点项目,国家自然科学基金,国家创新基金,博士点基金和部省科研基金项目300多项,其他横向科研课题350余项,国家发明特等奖1项,国家科技发明三等奖1项,部、省科技进步一、二等奖和其他奖项120余项。07年到帐科研经费1500多万元。近年来国家科技进步二等奖有:王耀南4项:智能图像信息处理方法及其在工业系统中的应用;高速灌装生产线智能检测分拣成套装备研制及其推广应用等。罗安2项:电能质量先进控制方法及工程应用;大型企业综合电气节能关键技术及应用;国家技术发明二等奖有:罗安一项:冶金特种大功率电源系统关键技术与装备及其应用。近2年获得IEEE第三大区唯一杰出工程师年度大奖1人,全国优秀科技工作者1人。

电气与信息工程学院传承“爱国务实、经世致用”的优良传统,积极参与“推进富民强省、争当五一先锋”竞赛,在教学、科研、学科建设等领域喜获全面丰收。一是人才培养卓有成效。学院确立人才培养的中心地位,创新人才培养成绩突出,近三年指导博士生获,“全国百篇优秀博士学位论文”获得者1人,省优秀博士论文5篇、省优秀硕士论文12篇、在全国“挑战杯” 大学生创业大赛上荣获金奖3人、全国大学生智能车竞赛上荣获一等奖6项;教学研究和教学改革成果显著,先后承担国家级教改项目1项、省部级教改项目5项。获国家级教学成果二等奖1项,省级教学成果一等奖1项,二等奖3项;工业自动化专业被评为国家特色专业,电气工程及其自动化专业被列为湖南省重点专业。二是科学研究硕果累累。学院围绕国家经济和国防建设重大需求,集中力量,以国家 “973”项目、“863”计划、国家重大科技专项、国家科技支撑计划、国家自然基金重点项目、湖南省重大专项为突破口,积极服务于国家和地方经济,实现了产学研的有机结合。形成了多个高水平科研团队,营造了良好的科研环境。近三年新增科研项目170项,科研项目到帐总经费8274万元,出版教材、学术专著15部,获得发明专利47项、软件著作权51项、实用新型专利33项,学术论文被三大检索收录479篇,荣获各类科技奖励共34项,其中国家科技进步二等奖4项,省部级奖励19项。三是学科建设成绩突出。学院已拥有十分完备的学科体系,拥有“控制理论与控制工程”国家重点学科,“电工理论与新技术”国家重点培育学科;拥有1个机器人感知与控制技术国家工程实验室,1个电能变换与控制国家工程技术研究中心,1个电力驱动与伺服技术国防重点学科实验室,2个教育部工程研究中心,1个教育部重点实验室,1个国防技术重点实验室,2个湖南省重点实验室,2个机械工业联合会重点实验室,2个博士后科研流动站,2个一级学科博士点,11个二级学科博士点,17个硕士点。四是综合管理井然有序。学院形成了“拼搏、奉献、和谐、快乐”的学院文化,制度健全、管理完善,院务公开、民主决策、勤政廉政,注重依靠教职工共商发展大计,凝心聚力,有关学院建设和发展的重大决策和涉及教职工切身利益的重大事项,都必须通过教职工代表大会的民主决策。学院呈现出持续快速发展的良好局面,党建与思政工作连续六年被评为学校优秀单位;部门工会连续七年被评为先进集体,2010年被授予湖南大学先进教职工之家称号,2011年被湖南省总工会授予“湖南省五一先锋集体”荣誉称号,2010年湖南省教育系统唯一获得该项荣誉的单位。